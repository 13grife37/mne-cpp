\section{Einleitung}

In den letzten Jahrzehnten nahm die Vernetzung von Computersystemen so sehr zu, dass sie mittlerweile zur Selbstverständlichkeit geworden ist.
Dieser Trend setzt sich auch heute noch fort und es werden immer größere Datenmengen versendet.
Insbesondere scheint die Verwendung einer zentralen elektronischen Datenverarbeitung essentiell für ein erfolgreiches Arbeiten in jeder Organisation zu sein. 

Da die Digitalisierung steigt, wächst jedoch auch die Abhängigkeit von den zugrundeliegenden Technologien. 
Bezogen auf Netzwerke bedeutet dies, dass neben der Verfügbarkeit besonderer Fokus auf die Vertraulichkeit und Authentizität gelegt werden muss.  
Durch die stetige Weiterentwicklung der Hardware wird die Verfügbarkeit immer besser beherrschbar. 
Auf der anderen Seite wird es immer schwieriger, Vertraulichkeit und Authentizität mithilfe eines Kryptosystems zu gewährleisten, wenn Daten zunehmend schneller übertragen werden sollen. 
Neben der reinen Bandbreite dürfen dabei auch Latenz und Jitter nicht signifikant beeinträchtigt werden.

Nur wenn ein Kryptosystem dies gewährleistet, wird es auch akzeptiert und schließlich eingesetzt. Bisher wird dazu jedoch teure oder inflexible Spezialhardware benötigt.

Durch die native Unterstützung von Verschlüsselung mithilfe des AES-Verfahrens auf modernen Prozessoren und neuer Schnittstellen zum direkten Zugriff auf Netzwerkkarten, wie sie das DPDK (Data Plane Developement Kit) bietet, haben sich neue Wege zur Entwicklung von Kryptosystemen geöffnet.

\textbf{PECTO} (Paket EnCryption on layer TwO) ist ein Framework, dass es ermöglicht, verschiedene zu schützende, rote Netze sicher zu verbinden. 
Da hier nichtkonforme Netzwerke mit unterschiedlichen Kommunikationsprotokollen arbeiten, agiert es als ein Gateway, welches die Kommunikation zwischen roten Netzen über ein unsicheres, schwarzes Netz ermöglicht. 
Dabei werden die Pakete, welche zwischen roten Netzen versandt werden, verschlüsselt über ein schwarzes Netz übertragen.
Im Gegensatz zu schon verbreiteten VPN-Lösungen erfolgt die Verschlüsselung jedoch auf Layer-2 des ISO/OSI-Schichtenmodells, behandelt also Ethernet-Frames statt IP-Pakete.

PECTO kann zur Absicherung von lokalen Netzwerken verwendet werden, ohne Spezialhardware nutzen zu müssen.
Es wird nur ein normaler x64-kompatibler Computer benötigt, auf dem ein Linux-basierendes Betriebssystem ausgeführt wird.