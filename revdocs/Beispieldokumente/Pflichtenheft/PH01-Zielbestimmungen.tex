\section{Zielbestimmungen}

An das zu entwickelnde Framework PECTO werden diverse Anforderungen gestellt.
Diese sind zunächst nach ihrer Wichtigkeit in Muss- und Wunschkriterien unterteilt. 

Zur Beschreibung der Anforderungen werden folgende Begriffe verwendet:
\begin{description}
	
	\item[Instanz] Unter einer Instanz versteht man ein aktives Element des Systems, welches den PECTO-Code auführt.
	Jede Instanz dient als Schnittstelle zwischen einem \gloss{roten} und dem \gloss{schwarzen Netz}.
	
	\item[Gruppe] Eine Menge mehrerer, verschiedener Instanzen, welche die gleiche rote Netzgruppe schützen, wird als Gruppe bezeichnet.
	Dabei besteht eine Netzgruppe logisch aus einem \gloss{roten Netz}, welches aber physisch durch ein \gloss{schwarzes Netz} in mehrere \gloss{rote Netze} geteilt wird. 
		
	\item[System] Der Begriff System wird als Synonym zum Namen des Projektes, PECTO, verwendet.

\end{description}

\subsection{Musskriterien}
Die Musskriterien bestimmen, welche Anforderungen an PECTO gestellt werden müssen, damit das System seine Funktion erfüllen kann.

\subsubsection{Verbindungsaufbau}
Um einen Verbindungsaufbau zwischen zwei Instanzen zu initiieren, müssen die folgenden Kriterien erfüllt sein.
	\begin{description}
	  \item[C1110] \textit{Authentisierung}: Zwei Instanzen einer Gruppe müssen einander identifizieren, indem sie sicherstellen, dass die jeweils andere Instanz dasselbe Kennwort zum Schutz der roten Netzgruppe kennt.

	  \item[C1120] \textit{Schlüsselaustausch für Unicast}: Zwei Instanzen einer Gruppe handeln einen Schlüssel aus, der zum Chiffrieren und Dechiffrieren der Pakete genutzt wird, welche zwischen den beiden Instanzen ausgetauscht werden. 
	  Dabei müssen folgende Sicherheitsanforderungen erfüllt sein: Authentizität(\textbf{C1420}), Vertraulichkeit (\textbf{C1430}), Perfect Forward Secrecy(\textbf{C1450}).
	  
	  \item[C1130] \textit{Schlüsselaustausch für Broad- und Multicast}: Alle Instanzen erhalten einen Gruppenschlüssel, welcher zuvor erstellt und verteilt wurde. 
	  Dieser Gruppenschlüssel wird zum Chiffrieren und Dechiffrieren derjenigen Pakete genutzt, die an alle Instanzen der Gruppe weitergeleitet werden.
	  Dabei müssen folgende Sicherheitsanforderungen erfüllt sein: Authentizität(\textbf{C1420}), Vertraulichkeit (\textbf{C1430}), Perfect Forward Secrecy(\textbf{C1450}).
	  	  
	 \item[C1140] \textit{{\color{glossb}Logging}}: Fehler beim Schlüsselaustausch müssen protokolliert werden, da diese auf einen Angriff auf das System hindeuten können.
	  
	\end{description}


\subsubsection{Forwarding bestimmter Pakettypen aus dem roten Netz}
Die folgenden Kriterien bestimmen, wie die Weiterleitung der Pakete aus einem \gloss{roten Netz} erfolgen soll. 
Dabei wird eine Unterscheidung bezüglich des Pakettyps vorgenommen.
	\begin{description}
	   \item[C1210] \textit{IPv4-Pakete (Unicast)}: Jede Instanz erhält durch eine Konfigurationsdatei Informationen darüber, wohin Pakete weitergeleitet werden müssen.
	   Für Unicast-Pakete wird anhand der IPv4-Destination-Address die Instanz gefunden, an die das Paket weitergeleitet werden muss. 
	   Es wird anschließend verschlüsselt und an genau diese Instanz übertragen.
	  
	   \item[C1220] \textit{IPv4-Pakete (Multi- und Broadcast)}: Multi- und Broadcast-Pakete werden chiffriert an alle Instanzen einer Gruppe verteilt. 
	   Die Verschlüsselung erfolgt hierbei mit dem Gruppenschlüssel so, dass dasselbe Paket nicht mehrfach verschlüsselt werden muss.
	  
	   \item[C1230] \textit{\gloss{ARP}-Pakete}: \gloss{ARP}-Pakete werden wie IPv4-Broadcast-Pakete (\textbf{C1220}) behandelt. 
	  
	   \item[C1240] \textit{Andere Protokolle}: Pakete, die nicht durch eine andere \textbf{C12**}-Regel behandelt werden, sind zu verwerfen. Insbesondere werden auch IPv6-Pakete verworfen.
	\end{description}

\subsubsection{Zugriffskontrolle}
Wie die Zugriffskontrolle zwischen den Netzen geregelt ist, bestimmen die folgenden Kriterien.
	\begin{description}
	   \item[C1310] \textit{Verschlüsselung}: Eine Instanz des Systems muss Pakete aus einem geschützten, \gloss{roten Netz} entgegennehmen, diese verschlüsseln und über ein ungeschütztes, \gloss{schwarzes Netz} an eine weitere Instanz weiterleiten. 
	   Der Header des Paketes soll erkenntlich machen, ob ein Paket von einer Instanz dieses Systems verschlüsselt wurde.	     
	  	  
	   \item[C1320] \textit{Entschlüsselung}: Wenn eine Instanz Pakete aus einem ungesicherten \gloss{schwarzen Netz} entgegennimmt und festgestellt werden kann, dass diese von einer anderen Instanz dieses Systems erzeugt wurden, muss es diese Pakete entschlüsseln. 
	   Konnte das Paket erfolgreich entschlüsselt werden, wird es in dem Netzwerk, in dem sich der Empfänger befindet, weitergeleitet.
	  
	   \item[C1330] \textit{Verarbeitung von Paketen aus dem \gloss{schwarzen Netz}}: Wenn ein Paket aus dem \gloss{schwarzen Netz} eine Instanz erreicht, wird überprüft, ob die im Paket-Header vorhandene Kennzeichnung mit der übereinstimmt, welche das System zur Verschlüsselung von Paketen nutzt.
	   Trifft dies nicht zu, verwirft die Instanz das erhaltene Paket.
	  	
	   \item[C1340] \textit{Kontrollierter Zugriff}: Das System isoliert die geschützte Netzgruppe vollständig, d.h. es dürfen keine Pakete zwischen Geräten im \gloss{schwarzen} und \gloss{roten Netz} ausgetauscht werden.
	  	  
	\end{description}


\subsubsection{Sicherheitsanforderungen}
	\begin{description}
	  \item[C1410] \textit{Integrität}: Instanzen stellen sicher, dass von einer anderen Instanz erzeugte Daten nicht modifiziert wurden. 
	  Durch Prüfsummen kann die Integrität der Daten festgestellt werden.  
	  
	  \item[C1420] \textit{Authentizität}: Durch ein Kennwort, dass den einzelnen Instanzen einer Gruppe bekannt ist, wird die Vertrauenswürdigkeit der anderen Instanzen derselben Gruppe überprüft.
	  
	  \item[C1430] \textit{Vertraulichkeit}: Instanzen stellen durch Verschlüsselung sicher, dass unautorisierte Dritte keinen Zugriff auf übertragene Nachrichten oder Informationen haben.
	  
	  \item[C1440] \textit{Verfügbarkeit}: Das System soll unempfindlich gegenüber Angriffen sein, welche darauf abzielen, die Stabilität und Lauffähigkeit des Systems zu beeinträchtigen. 
	  
	  \item[C1450] \textit{Perfect Forward Secrecy}: Die Kompromittierung eines Schlüssels, welcher zu einem früheren Zeitpunkt zur Aushandlung des eigentlichen Sitzungsschlüssels verwendet wurde, soll keinen Einfluss auf die Sicherheit der ausgetauschten Daten haben.
	  
	  \item[C1460] \textit{Infrastructure-Hiding}: Nach außen hin soll nicht erkennbar sein, welche und wie viele Einheiten innerhalb \gloss{roter Netze} miteinander kommunizieren.
	  Es ist nach außen nur sichtbar, welche Instanzen miteinander kommunizieren, nicht aber welche Einheit aus dem Rechnernetz von Instanz A mit welcher Einheit aus dem Rechnernetz von Instanz B. 
	\end{description}

\subsubsection{Nicht-funktionale Anforderungen an das System}
	\begin{description}
	
	  \item[C1510] \textit{\gloss{Robustheit}}: Der Austausch von Metadaten, insbesondere auch der Schlüsselaustausch, soll robust gegenüber Paketverlusten sein.	
	  	  
	  \item[C1520] \textit{Transparenz}: Der Einsatz des Systems darf keinerlei Änderungen der Konfigurationen der Netzwerkgeräte im \gloss{roten Netz} nötig machen.
	
	  \item[C1530] \textit{Skalierbarkeit}: Das System muss bis zu einer Anzahl von 100 Teilnetzen skalierbar sein, an die jeweils etwa 30 Endgeräte angeschlossen sind, ohne dass merkliche Performanceeinbußen auftreten.
	  
	  \item[C1540] \textit{Konfigurierbarkeit}: Die Instanzen des Systems müssen per Textdatei konfigurierbar sein.
	  
	  \item[C1550] \textit{Protokoll-Overhead}: Der Protokoll-Overhead soll eine Gesamtgröße von 50 Byte nicht überschreiten.
	\end{description}


\subsubsection{Kriterien an die Testumgebung}
	\begin{description}
	  \item[C1610] \textit{\gloss{Unit Tests}}: Alle Klassen, die nicht direkt vom \gloss{DPDK} abhängig sind, müssen \gloss{Unit-Tests} aufweisen.
	  
	  \item[C1620] \textit{Offline-Tests}: Die Verarbeitung von Paketen muss ohne Netzwerkartenanbindung auf ihre funktionalen und nicht-funktionalen Eigenschaften getestet werden können. 
	  
	  \item[C1630] \textit{Lasttests}: Die Gesamtperformance des Systems muss in einer realen Umgebung getestet werden.
	\end{description}


\subsection{Wunschkriterien}
Sofern die Musskriterien an das System erfüllt werden, ist es wünschenswert, folgende Kriterien umzusetzen.
	\begin{description}
	  \item[C2110] \textit{Automatischer Netzaufbau}: Eine Instanz des Systems kann in dem \gloss{schwarzen Netz} automatisch nach anderen Instanzen suchen, welche die gleiche Netzgruppenkennung haben. 
	  
	  \item[C2120] \textit{Rekeying}: Der Schlüssel zur Sicherung der Verbindungen zwischen einzelnen Instanzen kann regelmäßig erneuert werden. 
	  
	  \item[C2130] \textit{Statistiken}: Die Instanzen können Statistiken über die Auslastung, Paketeigenschaften und Fehlern führen.
	\end{description}
 

\subsection{Abgrenzungskriterien}
Die folgenden Kriterien grenzen die Funktionalitäten des Systems ab.
	\begin{description}
	  \item[C3110] Das System bietet den Schutz von Paketen auf Layer-2 des ISO/OSI Schichtenmodells.
	  Es erscheint daher nach außen wie ein \gloss{Layer-2-Switch}, der nur bestimmte Pakettypen weiterleitet. 
	  
	  \item[C3120] Zwischen zwei Instanzen darf kein Layer-3-Routing stattfinden.
	  Damit kann das System nur in Ethernet-Netzen eingesetzt werden und insbesondere keine Kommunikation schützen, die durch das Internet geleitet werden muss.
	   
	  \item[C3130] Die Hardware, auf der das System ausgeführt wird, muss eine \gloss{DPDK}-kompatible Netzwerkkarte und einen x64-kompatiblen Prozessor aufweisen, welcher \gloss{AES-NI} unterstützt.
	  Zudem muss ein Linux-basierendes Betriebssystem installiert sein.
	\end{description}

