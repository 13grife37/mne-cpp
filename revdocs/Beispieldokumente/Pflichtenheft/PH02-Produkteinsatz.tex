\section{Produkteinsatz}
Zur Beantwortung der Frage, was das System unter welchen Rahmenbedingungen leisten soll, werden Anwendungsbereich, Zielgruppe und Betriebsbedingungen spezifisch betrachtet.
\subsection{Anwendungsbereich}
Die Anwendung wird dort zum Einsatz kommen, wo der Transfer von Daten eine wichtige Voraussetzung für das Arbeiten an teilweise stark verstreuten Standorten darstellt.
Dabei spielt insbesondere die effiziente, verschlüsselte und sichere Übertragung, welche für riesige Datenmassen mit den herkömmlichen Methoden nur schwer umzusetzen ist, eine große Rolle.
Mit dem System wird versucht, die Verschlüsselung von Ethernet-Rahmen nahezu in Echtzeit zu ermöglichen.

\subsection{Zielgruppe}
Im Allgemeinen ist die gebotene Funktionalität für alle interessant, die große Massen an Daten, zwischen nicht direkt verbundenen Netzwerken über ein nicht vertrauenswürdiges Netz verschlüsselt übertragen wollen. 
Insbesondere kann dies für diverse Unternehmen, Behörden oder auch NGOs (Non-Governmental Organisations) nützlich sein, die organisationsintern große Datenmassen besonders schnell und sicher transferieren müssen.


\subsection{Betriebsbedingungen}
Um die Verschlüsselung von Ethernet-Rahmen in nahezu Echtzeit umzusetzen, kommt das \gloss{DPDK} zum Einsatz. 
Dieses ermöglicht Pakete direkt in den Hauptspeicher des Rechners zu übertragen, ohne Interruptlast zu erzeugen.
Für die Umsetzung des Verschlüsselungsverfahrens verwendet das System \gloss{AES}, wobei die Verschlüsselung durch Zuhilfenahme von \gloss{AES-NI} beschleunigt wird.
Bei langfristiger Nutzung des Systems wird versucht hohe Stabilität zu gewährleisten, damit eine wartungsfreie Laufzeit ermöglicht wird.
