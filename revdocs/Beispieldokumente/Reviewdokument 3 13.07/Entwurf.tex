\section{Ziele im beim Entwurf PECTOs}
An das zu entwickelnde Framework PECTO stellten wir verschiedene Anforderungen, insbesondere auch an die Performance und Sicherheit. 
Diese finden sich auch an unterschiedlichen Stellen in den Entwurfsentscheidungen wieder.  
Dabei konkurriert der Sicherheitsgedanke, welcher nach möglichst einfachem Code strebt, stets mit dem Wunsch nach hoher Performance.
Um einen guten Ausgleich zu finden, werden alle Komponenten zunächst auf gute Verständlichkeit des Codes optimiert.
Sollten sich beim Profiling Probleme zeigen, werden diese Stellen gezielt überarbeitet.

\subsection*{Verwendete Muster}
Zur Realisierung von PECTO werden folgende Entwurfsmuster verwendet:

\begin{description}
	\item[Inversion of Control (IoC)] 
	Um das Schichtenmodell umsetzen zu können und zyklische Abhängigkeiten zu vermeiden, wird Inversion of Control eingesetzt. 
	Die tiefere Schicht stellt dafür jeweils Interfaces bereit, die von den höheren Schichten implementiert werden. 
	Der sich dadurch ergebende Overhead ist in der Regel zu vernachlässigen.
	Insbesondere überwiegen die Vorteile durch eine bessere Entkoppelung.
	Wenn sich Performance-Probleme ergeben, kann davon abgewichen werden.
	
	\item[Strategy Pattern]
	An verschiedenen Stellen im Entwurf können verschiedene Algorithmen verwendet werden. 
	Damit dies konfgurierbar ist, wird das Strategy Pattern verwendet.
	Ein Beispiel hierfür ist die Wahl des Verschlüsselungsalgorithmus, der zu Debuggingzwecken und für Erweiterbarkeit verschiedenartig gewählt werden kann.
	Zur Auswahl stehen zum Beispiel AES-GCM, eine unsichere XOR-Verschlüsselung oder gar keine Verschlüsselung. 
	
	\item[Exceptions]
	Zur Vereinfachung des Codes werden Exceptions eingesetzt.
	Diese sind immer dann auszugeben, wenn ein unerwartetes Ereignis eingetreten ist.
	Da der Einsatz von Exceptions einen erheblichen Overhead zur Folge haben kann, wenn zu viele von ihnen geworfen werden, müssen für besonders performance-kritische Funktionen immer zwei Schnittstellen zur Verfügung stehen.
	Die erste dieser Schnittstellen löst die Exceptions aus, während die zweite über Rückgabewerte entscheidet ob die Funktion erfolgreich ausgeführt wurde.
	
	\item[Code Contracts]
	Jede Methode, die zu einer öffentlichen Schnittstelle gehört, validiert ihre Parameter und Ausgaben.
	Unerwartete Eingaben, beispielsweise Nullzeiger, müssen erkannt werden und zu einer Exception führen.
	Ebenfalls sollen alle Ausgaben einer einfachen Plausibilitätsprüfung unterzogen werden. 
	Auch hier liegt besonderes Augenmerk auf Nullzeigern.
	So wird sichergestellt, dass keine unerwarteten Rückgabewerte zu Fehlern an anderer Stelle führen können.
	Ausnahmen bilden einige performancekritische Funktionen, die in auch eine Variante bereitstellen, die keine Prüfung durchführt.
\end{description}
