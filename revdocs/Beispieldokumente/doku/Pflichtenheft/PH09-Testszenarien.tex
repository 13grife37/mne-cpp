\section{Testszenarien und Testfälle}


Die hier aufgeführten Testfälle funktionaler Eigenschaften können überwiegend mittels Komponententests durchgeführt werden. Diese Tests werden über das {\color{glossb}DPDK} und in C++ über das Framework {\color{glossb}cxxtests} erstellt und müssen auf den Rechnern der Entwickler ausgeführt werden. Anschließend wird das Gesamtsystem getestet.


\subsection{ Funktionaler Eigenschaften}

	\begin{description}
		\item[T110] \textit{Funktion der Schnittstellen:} Das Testen der Schnittstellenfunktionalität zwischen {\color{glossb}DPDK} und der {\color{glossb}Network Abstraction-Komponente} wird manuell durchgeführt.
		
		
		\item[T120] \textit{Schnelle Verarbeitung:} Eine erzeugte {\color{glossb}Dispatch-Komponente} wird bezüglich dessen Zeitperformance auf korrektes Multi-Threading überprüft.
		
		
		\item[T130] {\color{glossb}Unit-Tests} zu dem {\color{glossb}CPH}-Element werden mittels Hilfsobjekten, die den Schlüsselaustausch testzwecks ersetzen sollen ermöglicht. Diese sind als {\color{glossb}Mocks} realisiert.
		
		\item[T140] \textit{Transportweg:} Das Forwarding wird durch automatisiert erzeugte unterschiedliche Pakete auf richtige Weiterleitung getestet (\textbf{F410}-\textbf{F440}).
		
		
		\item[T150] \textit{Schlüsselaustausch:} Hier sollte die korrekte Übertragung und Verknüpfung der beiden Schlüssel, sowie die Bedingungen der Perfect Forward Secrecy (\textbf{C1450}) gewährleistet sein.
		
		
		\item[T160] \textit{Verschlüsselung:} Im {\color{glossb}AES}-Teil werden zuerst die einzelnen Klassen untereinander und anschließend die vollständige Codierung/Decodierung (\textbf{F310}-\textbf{F320}) der Pakete auf Konsistenz getestet. 
	\end{description}



\subsection{Testfälle nicht-funktionaler Eigenschaften}


Tests zu den nichtfunktionalen Eigenschaften werden in einer virtualisierten und einer realen Testumgebung durchgeführt.


\subsubsection{Emulierte Testumgebung}

	\begin{itemize}
		\item Ausführbar auf einzelnem Rechner 
		\item \textit{Multi-User-Test:} Simulation von Gruppenkommunikation mit vielen Benutzern (siehe \textbf{C1170})
		\item Erzeugter Traffic muss die unter {\color{glossb}Robustheit} \textbf{L040} aufgeführten Eigenschaften aufweisen.
	\end{itemize}


\subsubsection{Reale Testumgebung}

	\begin{itemize}
		\item Durchführung auf mehreren Rechnern
		\item Kommunikation im lokalen Netz des Labors
	\end{itemize}

\subsubsection{Robustheit}

Zum Testen der {\color{glossb}Robustheit} (siehe \textbf{L050}) wird ein Stresstest in Kombination mit einem Chrashtest durchgeführt.
Hierfür muss während des Schlüsselaustausches eine Ausnahmesituation hoher Verlustraten und hoher Paketverzögerung hergestellt werden. 
Trotz auftreten einer solchen Situation, darf dann keine Überlastung auftreten.


\subsubsection{Sicherheit}

Um das System auf potentielle Sicherheitslücken zu testen, müssen konkrete Angriffsszenarien wie das Überlasten des Systems und die Zugabe von manipulierten Paketen von außen simuliert werden (siehe \textbf{L010}).


\subsubsection{Speichereffizienz}

Während der anderen Tests ist zu beobachten, ob der benötigte Speicher im gewünschtem Rahmen bleibt (siehe \textbf{L060}).


\subsubsection{Recheneffizienz}

Die Recheneffizienz wird bei unterschiedlicher Auslastung durch analysieren der Ruhe und Arbeitszeiten einer Momentaufnahme kontrolliert (siehe \textbf{L070}).