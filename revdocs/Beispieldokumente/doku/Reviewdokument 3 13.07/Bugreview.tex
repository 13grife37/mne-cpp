\section{Bugreview}
Im folgenden werden die schwerwiegendsten Probleme beschrieben, welche während der Erstellung dieser Software aufgetreten sind. 
Außerdem wird dargestellt wie diese gelöst werden konnten.

\subsection{Receive Side Scaling (RSS) im schwarzen Netz}
In diesem Abschnitt wird das Receive Side Scaling beschrieben.
Dabei handelt es sich um die verwendete Technologie, um Pakete zwischen verschiedenen Warteschlangen zu verteilen.

\begin{description}
	
	\item[Receive Side Scaling] 
	Receive Side Scaling ist eine Technologie, bei der die Netzwerkkarte empfangene Pakete auf verschiedene Warteschlangen verteilt.
	Diese wird von PECTO benutzt, um das Load Balancing zwischen verschiedenen Threads umzusetzen.
	
	\item[Problembeschreibung]
	RSS nutzt verschiedene Informationen aus höheren Schichten, z.B. IP-Adressen, um die Pakete sinnvoll an verschiedende Warteschlangen zu verteilen. 
	Wenn jedoch ein PECTO-Paket aus dem schwarzen Netz empfangen wird, kann auf diese Informationen zwangsläufig noch nicht zugegriffen werden.
	Damit kann auch kein Balancing stattfinden.
	
	\item[Workaround]
	Einige teurere Netzwerkkarten können RSS auch anhand des gesamten Paketes ausführen, jedoch ist dies keine Lösungsmöglichkeit, welche für die Realisierung dieses Systems in Frage kommt.
	Ein anderer Workaround ist das Setzen von benutzerdefinierten Filtern, sogenannten Flex Filters, auf dem Netzwerkinterface, die verschiedene Pakete explizit den verschiedenen Queues zuordnen.
	
	\item[Problemlösung]
	Es wurde ein Feld zum manuellen Round-Robin-Loadbalancing in die PECTO-Header eingefügt. 
	Mithilfe meherer Flex-Filter werden die Pakete dann automatisiert von der Netzwerkkarte auf mehrere Threads verteilt.
	Somit wird kein RSS mehr verwendet, sondern eine gleichweritge Alternative.
	 
\end{description}
