\section*{Reviewdokument der Validierungsphase}
In diesem Reviewdokument sollen die Ergebnisse des Softwareprojektes in der dritten Phase betrachtet werden.
Insbesondere wird auf wichtige Testfälle eingegangen, die beim Testen der Software entwickelt wurden.
Als Anhang liegt die komplette von Doxygen generierte Entwicklerdokumentation und ein Benutzerhandbuch, welches die Installationsanleitung beinhaltet, bei.


\section{Ziele und Relevanz des Frameworks PECTO}
PECTO (Paket EnCryption on layer TwO) ist ein Framework, dass es ermöglicht, verschiedene zu schützende, \gloss{rote Netze} sicher zu verbinden.  
Da hier verschiedenartig sturkturierte Netzwerke mit unterschiedlichen Kommunikationsprotokollen arbeiten, agiert es als ein Gateway, welches die Kommunikation zwischen roten Netzen über ein unsicheres, \gloss{schwarzes Netz} ermöglicht. 
Dabei werden die Pakete, welche zwischen roten Netzen versandt werden, verschlüsselt über ein \gloss{schwarzes Netz} übertragen.
Im Gegensatz zu schon verbreiteten \gloss{VPN}-Lösungen erfolgt die Verschlüsselung jedoch auf Layer-2 des ISO/OSI-Schichtenmodells, behandelt also Ethernet-Frames statt IP-Pakete.
PECTO kann zur Absicherung von lokalen Netzwerken verwendet werden, ohne Spezialhardware nutzen zu müssen.