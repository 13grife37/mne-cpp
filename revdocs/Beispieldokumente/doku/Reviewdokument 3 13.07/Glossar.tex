\section{Glossar/ Abkürzungen}

\begin{description}
	\item[AES] (Advanced Encryption Standard) ist ein deterministisches Verschlüsselungsverfahren, bei dem durch einen Schlüssel ein Text fester Länge in ein Chiffre fester Länge transformiert wird. 
	
	\item[AES-NI] (Advanced Encryption Standard New Instructions) ist eine Erweiterung zur x86-Befehlssatzarchitektur für Mikroprozessoren von Intel und AMD. Man kann hiermit eine Verbesserung der Geschwindigkeit von Anwendungen, welche AES-Ver- und Entschlüsselungen nutzen, erzielen. 
	
	\item[ARP] (Address Resolution Protocol) ist ein Protokoll, mit dem Netzwerkadressen auf Hardwareadressen abgebildet werden können, damit eine Kommunikation auf dem Network Layer stattfinden kann.  
	
	\item[Chiffre] ist ein Geheimtext, der unter Verwendung eines Schlüssels mit kryptographischen Verfahren derart verändert wurde, dass es nicht mehr möglich ist, dessen Inhalt zu verstehen.
	
	\item[CPH] (Control Paket Hub) regelt die Verarbeitung von Schlüsselpaketen innerhalb PECTOs.
	
	\item[cxxtests] ist ein Framework, welches zur Erstellung von Unit-Tests verwendet wird.
	
	\item[Dispatch-Komponente] steuert die Einteilung der Pakete (verschlüsselt/unverschlüsselt) für das System.
	
	\item[DPDK] (Data Plane Development Kit) ist eine Sammlung von Bibliotheken und Netzwerkkontrolltreibern, die zur schnellen Paketverarbeitung genutzt werden kann.
	
	\item[EAL] (Environment Abstraction Layer) ist eine Hardwareabstraktionsschicht, die erzeugt wird, um direkte Anfragen an die Hardware leichter zu stellen und die allgemeine Nutzung zu vereinfachen.
		
	\item[Effizienz] ist das Ausmaß der Sparsamkeit des Systems bezüglich seiner Ressourcen. Ziel sind insbesondere ein geringer Speicherverbrauch, eine geringe CPU-Last und eine hohe Paketrate.
	
	\item[IV-Space] ist der separiete Zahlenraum, welcher jeder Instanz des Systems individuell zugeordnet wird, um unterschiedliche Initialisierungsvektoren zu erstellen.
	
	\item[Layer-2-Switch] ist ein einfaches Kopplungsgerät, das lokale Netzwerksegmente miteinander verbindet und eine Weiterleitfunktion der Datenpakete, auf dem Data Link Layer, übernimmt. 
	Sie haben insbesondere keine Vermittlungs- und Routingfunktionen.  
	
	\item[Logging] ist das automatische Speichern von Datenänderungen, welche in Logdateien hinterlegt werden.
	
	\item[Mock] ist ein Objekt, welches das Verhalten eines realen Objektes nachbildet, und für Unit-Tests verwendet wird.
	
	\item[Network Abstraction-Komponente] abstrahiert die Verwendung des DPDK und bildet die Schnittstelle zum übrigen System.
	
	\item[Paketdurchsatz] ist die Anzahl der Pakete, die in einer bestimmten Zeit gesendet werden können.
	
	\item[Passphrase] ist eine Zeichenfolge, über die der Zugriff auf ein Netzwerk gesteuert wird.
	
	\item[Portabilität] ist die Möglichkeit das System auf einem anderen Betriebssystem einzusetzen.
	
	\item[Robustheit] ist die Fähigkeit, auch unter ungünstigen Bedingungen zuverlässig zu funktionieren. Sie dürfen zu keinerlei Problemen führen.
	
	\item[Sicherheit] ist die Fähigkeit, dass Systemfunktionen nicht von einer dritten Person abgehört oder manipuliert werden können.
	
	\item[Skalierbarkeit] ist die Fähigkeit eines Systems, die Leistung durch das Hinzufügen von Ressourcen zu steigern.
	
	\item[Unit-Test] ist ein Test, der verwendet wird, um Einzelteile von Computerprogrammen auf korrekte Funktionalität zu testen.
	
	\item[Zuverlässigkeit] ist die Fähigkeit, dass ein Programm während einer gewissen Betriebsdauer nur begrenzt viele Fehlerfälle aufweisen darf.
	
	\item[Polling] bezeichnet in der Informatik die Methode, den Status eines Geräts aus Hard- oder Software oder das Ereignis einer Wertänderung mittels zyklischem Abfragen zu ermitteln.
	
	\item[GCM] ist ein Betriebsmodus, in der Blockchiffren für eine symmetrische Verschlüsselungsanwendung betrieben werden können
	
	\item[Bug] bezeichnet im Allgemeinen ein Fehlverhalten von Computerprogrammen.
	
	\item[VPN] steht für Virtual Private Network. Dient dazu, Teilnehmer des bestehenden Kommunikationsnetzes an ein anderes Netz zu binden.
	
	\item[Memory Pool] ist ein dynamischer Speicher mit festen Blockgrößen.                                     
	
	  
\end{description}