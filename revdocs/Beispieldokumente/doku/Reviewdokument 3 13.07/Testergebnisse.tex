\section{Testergebnisse}
Um die Funktionalität und Korrektheit unserer Software zu überprüfen, wurden einige Unit Tests und manuelle Tests durchgeführt. 


\subsection{Überprüfung des korrekten Verhaltens wichtiger Komponenten}
Die Unit Tests aller getesteten Kompontenen verlaufen erfolgreich. 
Für die verschiednenen Module wurden folgende Eigenschaften geprüft:

\begin{description}
\item[PESpinlock]
	Der gegenseitige Ausschluss wird getestet, indem zwei Threads gleichzeitig die gleiche Variable hochzählen.
	
\item[PEMemoryBuffer]
	Das Anfügen und Löschen von Daten am Anfang und Ende wird getestet. 
	Zudem wird überprüft, ob Puffer nach ihrer Verwendung wieder korrekt freigegeben werden.
	
\item[PEForwardingTable]
	Die korrekte Übersetzung von IP-Addressen in MAC-Addressen der zuständigen Instanz wird überprüft.
	
\item[PECryptoAlgorithm]
	Die verschiedenen Verschlüsselungsmethoden werden auf ihre Performance und Funktionalität getestet. 
\end{description}

Weitere Komponenten wurden nicht durch Unit Tests überprüft, weil dazu umfangreiche Fakes und Mocks nötig wären.

\subsection{Nachweis funktionaler und nichtfunktionaler Eigenschaften durch manuelle Tests}
Verschiedene manuelle Tests wurden genutzt, um die funktionalen und nichtfunktionalen Eigenschaften nachzuweisen.
Die Tests fanden in einer realen und einer virtualisierten Umgebung statt.

Grundsätzlich wurden zwei Arten von Tests vorgenommen: Lasttests und Funktionstests.
Die Lasttests fanden stets in der nichtvirtualisierten Umgebung statt, um aussagekräfigte Ergebnisse zu liefern. Funktionstests wurden in beiden Testumgebungen ausgeführt.

Folgende Funktionstests wurden ausgeführt:

\begin{description}
\item[Ping-Test]
	Beim Ping-Test wurden Pakete von verschiedenen roten Teilnetzen aus in jeweils andere Teilnetze versendet. 
	Dadurch konnte die Funktionalität von ARP- und IP-Weiterleitung getestet werden. 
	Das Gesamtnetz hat dabei die Netzadresse 10.0.0.0/16, während die einzelnen roten Teilnetze die Form 10.0.$x$.0/24 besaßen.
	In jedem Teilnetz gab es genau einen Teilnehmer, welcher die IP-Adresse 10.0.$x$.1 verwendete.
	
	Die Anzahl der Teilnetze betrug in der realen Testumgebung zwei ($x \in \{1, 2\}$) und in der virtualisierten Umgebung drei ($x \in \{1, 2, 3\}$).

\item[TCP-Übertragungstest]
	Mithilfe des Linux-Tools \texttt{iperf} konnte die Funktionalität von TCP-Verbindungen und die Übertragung von großen Paketen getestet werden.
	Dabei konnten auch Probleme entdeckt werden, welche Folge einer zu klein gewählten MTU waren.
	
	Die Netze wurden wie beim Ping-Test gewählt.
\end{description}

Neben den Funktionstests wurden in der realen Umgebeung auch Performancetests ausgeführt.
Hierbei gab es zwei Arten von Tests. 
Einerseits wurde ein Netz, wie es schon beim Ping-Test beschrieben ist, verwendet, andererseits wurden noch einfachere Netze genutzt, um die Performance der Kryptokompontenten im Gesamtsysten zu testen.
Das reine Performancetestnetzwerk besteht aus zwei Clients mit den IP's 1.1.1.1 und 2.2.2.2.
Beide senden und empfangen mithilfe eines PECTO-basierenden Tools kleine IP-Pakete.
So kann die Performance des Systems bei der Verarbeitung von 64 Byte großen Paketen untersucht werden.

Die einzelnen Tests lieferten Folgende Ergebnisse:

\begin{description}
\item[Lasttest mit kleinen Paketen]
	Das Generator-Tool produziert 1,43 Mio. Pakete pro Sekunde, wovon wieder 1,01 Mio. Pakete auf der Gegenseite empfangen werden. 
	Die entstehenden rund 30\% Verlust kommen dadurch zustande, dass der Generator die Gigabit-Leitung voll auslastet, die verschlüsselten Pakete jedoch größer sind und damit die vorhandene Bandbreite überstiegen wird.

\item[Latenzbestimmung]
	Der Aufbau für die Latenzbestimmung entspricht dem Ping-Test im realen Netz.
	Es werden etwa 1000 Ping-Pakete pro Sekunde versendet und die Round-Trip-Time gemessen. 
	Durch eine Verlgeichsmessung ohne zwischengeschaltene PECTO-Instanzen kann die Latenz der Ver- und Entschlüsselung abgeschätzt werden.
	
	Die Auswertung dieses Tests ergab, dass die Latenz für eine Verschlüsselung und anschließende Entschlüsselung kleiner Pakete etwa 65-70\si{\micro\second} beträgt.
	Diese Verzögerungen können jedoch nur eingehalten werden, wenn das System nicht ausgelastet ist.

\item[Durchsatzbestimmung für TCP-Verbindungen]
	Die Durchsatzbestimmung erfolgte mit demselben Versuchsaufbau wie der TCP-Übertragungstest in der realen Umgebung. 
	Es ergab sich ein Durchsatz von 913 MBit, wobei der limitierende Faktor wiederum die fehlende Bandbreite im schwarzen Netz war.
	
\end{description}

  