\section{Requirements} 

	The product receives EEG/MEG sensor data and constructs a real-time 3D visualization of the brains current activity.
	Users can choose between further options, changing the output immediately to their personal preferences. 

\subsection{Mandatory criteria}

	The following functions have to be implemented correctly and must fulfill given requirements.
		
\subsubsection{Surface Constrained Distance Calculation (Geodesic problem on meshes)} \label{scdc}
	
	\begin{aims}
		
		\item[\textbf{\textit{Note:}}] From this point on \textbf{S}urface \textbf{C}onstrained \textbf{D}istance 
									   \textbf{C}alculation will be referred to as \textbf{SCDC}.
	
	\end{aims}
	
	Because the brain has an uneven surface, a function for calculating the distance between two separate points is needed.
	
	As the euclidian distance would not respect the structure of the surface, a different approach for determining the exact 	distances has to be implemented.  
	The function receives input data in form of a preprocessed triangulated surface mesh and calculates the distance between 	the vertices.			

	\begin{aims}
	
		\item[C111] Based on that data, the function calculates a matrix that holds values describing the distances between 						all vertices using double precision. 
		\item[C112] The function must be able to process up to 200,000 vertices.
		\item[C113] The user can limit the calculation to a subset of vertices. 
	
	\end{aims}

\subsubsection{Point to plane mapping} \label{projection}

	Since the sensors do not directly touch the head, therefore float slightly above it, an accurate projection is needed to 
	exactly localize their positions regarding the brain. 
	
	Hence a function must be implemented to solve this problem. 
	The function receives a set of sensor locations in 3D-Space and maps them onto the underlying surface mesh. Thus every 			sensor gets assigned to a vertex of the mesh. 

	\begin{aims}
	
		\item[C121] The function must be able to handle data from MEG-sensors which have a known orientation.
		\item[C122] The function must be able to handle data from EEG-sensors which are non-orientated.
	
	\end{aims}

\subsubsection{Interpolation algorithm} \label{interpolation} 

	The core feature is the ongoing interpolation, visualizing a particular set of sensor data, representing the brains 			activity.
	
	Because the number of vertices is bigger than the quantity of sensor points, most vertex-values must be interpolated.
	Thus the algorithm receives a mesh and a subset of vertices with their respective sensor data.  
	
	\begin{aims}
	
		\item[C131] Based on the said subset the algorithm must calculate the values for every vertex of the mesh.
		\item[C132] For this, the algorithm creates a matrix storing weights for the later interpolation.
					The interpolation process can be summarized by the following equation: 
					
					$y_{full} = W \cdot y_{sub}$
					, where $W$ is the mentioned matrix and $y_{sub}$ is the current dataset for the known sensors, i.e. 							vertices.
		\item[C133] The calculation of the weight matrix must be based on the result of the SCDC (\ref{scdc}).
		\item[C134] Bad channels, a part of the given sensor data, must be considered during processing. 
	
	\end{aims}

\subsubsection{Integration in Disp3D} \label{integration}
	
	In order to ensure usability within the given framework MNE-CPP, the final visualization must be integrated into the 			preexisting GUI, namely Disp3D.
	
	\begin{aims}
		
		\item[C141] A new function must be added to the Disp3D tree model. Internally this function must create a new 								handler. 
		
		%TODO in scan einfuegen nicht gleich disp3D
	\end{aims}
	
\subsubsection{Non-Functional Requirements}		
	
	%TODO Graka-Matrizen-Shice
	
	\begin{aims}

		\item[C151] The software has to run on the latest versions of 2 operating systems, namely Windows and Linux. 
		%TODO vll nur Ubuntu?
		%TODO ueberlegen, ob das vielleicht besser nur in Product environment steht
		\item[C152] Features \ref{scdc} and \ref{projection} are implemented in the class \textit{GeometryInfo}, while 								\ref{interpolation} is facilitated in the class \textit{Interpolation}.
		\item[C153] The function to integrate the product into Disp3D (\ref{integration}) is named \textit{addSensorData()}.
		\item[C154] For introduction purposes, a product video is to be created and published on the MNE-CPP website.  
		%TODO change name of addSensorData() like said in the meeting 26/04/2017
	\end{aims}
	
\newpage	
	
\subsection{Optional criteria}
	
	As long as the mandatory requirements are fulfilled, it is desirable to further match the following criteria. 	
	
\subsubsection{SCDC (\ref{scdc})}
	
	\begin{aims}
		
		\item[C211] The computation time should not exceed 1 second.
		%TODO change as of meeting 26/04/2017 (1 second may not be achievable, perhaps calculation using a threshold)
			
	\end{aims}
	
\subsubsection{Interpolation (\ref{interpolation})}

	\begin{aims}
	
		\item[C221] One interpolation cycle should take less than 17ms.
		\item[C222] Multiple methods for calculating the weight matrix can be implemented. The user can select one.
		\item[C223] The Computation is executed on GPU-level.
	
	\end{aims}
	
	
\subsection{Delimiting criteria} %TODO besserer Titel wünschenswert
	
	The following criteria limit the functionality of the system.
	
	\begin{aims}
		
		\item[C311] The program receives preprocessed data and does not get in touch with hardware sensors.
		\item[C312] The program does not evaluate the data medically and solely processes the data for further 										visualization. 						
		
	\end{aims}
	



	
