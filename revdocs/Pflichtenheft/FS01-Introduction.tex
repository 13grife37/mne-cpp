\pagenumbering{arabic}

\section{Introduction}

	Medicine today is highly advanced and is able to treat an immense amount of diseases and disorders.
	This results in high standards and a lot of pressure on people working in this branch. In order to further improve the 			standards the amount of errors has to be minimized, since they can have fatal consequences.

	In pursuance of achieving, maintaining as well as improving these abilities, technological assistance is of utmost 				importance.\\  

	Since the brain is one of the main organs, caution and accuracy is essential while treating its conditions. 
	
	Techniques such as the MEG (Magnetoencephalography) or EEG (Electroencephalography) observe the brains activity by 				measuring and monitoring magnetic fields or electrical deviations.

	These procedures help diagnosing epilepsy, migraine variants and other brain diseases. In addition, assistance in 				identifying brain death is one possible usage of EEG/MEG.
	 
	Furthermore they are used for research in fields like psychology.
	A proper visualization aids the usability of generated data and provides a superficial graphic overview of the brains 			activity, thus enabling first interpretations or even diagnosis.\\
	

	The MNE-CPP  project builds tools for the purpose of making the analysis of EEG and MEG data easier.
	Thus, the whole project is open-source and everyone can contribute. As programming language only C++ is used, although the project 			simultaneously exists in other languages, e.g. Python. 
	MNE-Scan, -Analyze and -Browse are some of the features of the existing framework. \\

	The new extension of the current project focuses on real-time 3D-visualization of EEG/MEG sensor data, while being as 		exact and fast as possible. It should both be integrated into the existing code and accessible through the MNE-Client as well as 			compatible with operating systems Linux and Windows.
  
