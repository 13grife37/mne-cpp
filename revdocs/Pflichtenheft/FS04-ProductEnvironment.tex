\section {Product Environment}
\subsection{Scenario}
As mentioned before, the aim of the product is to visualize MEG and EEG data in real-time. The software should reconstruct actual brain activity based on given sensor signals.
The two possible sources of data are:
\begin{list}{$\cdot$}{}

\item The MEG, short for Magnetoencephalography, is a measurement of the neurophysiological activities of the brain. The changes of the magnetic field are recorded by sensors which are non-invasive (on the outside of the head). Besides research, the MEG is used for planning complex brain-surgeries.

\item The EEG, short for Electroencephalography, is a measurement method for recording the electrical activities on the surface of the head. Electrodes that directly touch the skin measure potential fluctuations (brain waves) and amplify them. It is a standard method in neurology.
\end{list}

\subsection{Software}
\subsubsection{The MNE-CPP Framework}The MNE-CPP framework provides several applications, dividable into front-end and back-end software. MNE Scan, MNE Browse and MNE Analyze are front-end applications which are designed for high usability.
Beneath the front-end is the library layer which contains the core libraries provided by the framework.\\
The application must be integrated into the library layer.
\subsubsection{System}
The software must be able to run on both Windows and Linux in order to be accessible for more users. It requires OpenGL compatibility and depends on the Qt libraries (e.g. Qt3D).
\subsection{Hardware}
The program should run on most modern computers with enough performance to handle it. \\
A BCI-system (e.g. Elekta Neuromag® Vector View™) is connected to the computer and sends data which is handled by the software.