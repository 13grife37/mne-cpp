\section{Product Functions}

\subsection{User Functions}

	\begin{aims}
	
		\item[F11] The user can access the implemented features as a part of MNE Scan (see section \ref{3_1_scope}).
		\item[F12] Either MEG or EEG data can be used as input.
		\item[F13] Any desired surface mesh and set of sensor data, selected by the user, can be used as input data for 							further calculations.
		\item[F14] A preferred subset of vertices can be selected, to perform distance calculations on. 
		\item[F15] The user is able to select a threshold for identifying all relevant vertices while interpolating. 
		\item[F16] As part of MNE Scan/Disp3D all graphical options are available to the user, meaning e.g. different 								coloring and rotation of the object.
	
	\end{aims}

\subsection{Offline Functions}
	
	These functions are performed one time and are executed prior to the interpolation process.	
	
	\begin{aims}
	
		\item[F21]	The distance between two vertices on a given mesh, is calculated following the structure of the surface. 						Therefore the euclidian distance is not used.
		\item[F22] All floating sensor points are projected onto the nearest vertices on the input mesh. 
		\item[F23] A weight matrix, containing values for all vertices, is generated. 
 	
	\end{aims}
	
\subsection{Real-Time Functions}

	These functions are performed continuously and in real-time.	
	
	\begin{aims}
	
		\item[F31]	Based off prior calculations, the interpolation assigns every vertex a value.
		%\item[F32] A visual output is generated. %TODO bleibt das so?
	
	\end{aims}