\section{Test Scenarios}

\subsection{Functional Testing}

\subsubsection{Projection Testing}

To verify the sensor-mesh-mapping algorithm in case of EEG-sensors, the closest vertex to a
sensor can be determined by calculating the euclidean distance between said sensor and each
vertex of the mesh and comparing the minimal result of this calculation with the outcome of
the projecting algorithm. In case of MEG-sensors, the algorithm can be tested by comparing
both euclidean distance and solid angle between each vertex and the sensor, i.e. the sensors
location combined with its orientation.

\subsection{Nonfunctional Testing}

\subsubsection{Computation Time Testing}

In order to identify time consuming sections of code, the passed time is recorded for every
major step within the respective features of the program. This helps to ensure a high level of
efficiency.

\subsubsection{Memory Allocation Testing}

Since the program is to be run alongside an ongoing MEG/EEG-scan, it must not take up
too much memory. To assure this, the program should be profiled with suitable tools during
runtime.

\subsubsection{Pattern Testing}

As there already are more or less accurate implementations of the interpolation, the general
pattern of brain activity during one example can be recorded using these preexisting tools. Later, this can be compared to the output of the real-time mesh utilities.