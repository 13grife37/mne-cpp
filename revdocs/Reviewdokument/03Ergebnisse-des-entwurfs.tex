\section{Ergebnisse des Entwurfs}

Die Ergebnisse des Entwurfs sind zum größten Teil in der Entwurfsdokumentation nachzulesen.
Dort sind die Zusammenhänge der verschiedenen Pakete, Komponenten und Klassen aufgezeigt und anschaulich mit Hilfe von UML-Diagrammen dargestellt.

\subsection{Werkzeuge} 

Die verwendeten Werkzeuge sind Softwarelösungen, die die einzelnen Bereiche der Organisation und Entwicklung ermöglichen bzw. erleichtern.

\subsubsection{Organisatorische Werkzeuge}
	\begin{description}
		\item[Phabrikator:] Zuweisung einzelner, zu bearbeitender Tasks und deren Verwaltung, sowie die Bereitstellung des Wiki's. 
		
		\item[Quellcodeverwaltung:]	Hierfür wird \textit{Subversion} benutzt um einen einheitliche Arbeitsgrundlage an den Dateien zu gewährleisten.
		
		\item[Arbeitszeitenerfassung:] Hierfür wird \textit{Kimai} verwendet, dort werden alle Zeiten eingetragen und verschiedenen Themen/Bereichen zugeteilt.
		
		\item[LaTeX:] Hierbei handelt es sich um eine Textbeschreibungssprache, mit der verschiedenste Dokumente erstellt werden können. 
		Diese stehen nach der Erstellung in verschiedenen Formaten zur Verfügung.
		Im speziellen wird es in diesem Projekt für die Erstellung des Pflichtenheftes, den Reviewdokumenten und dem Entwurfsdokument verwendet.   
		
		\item[Doxygen:] Mit diesem Programm werden die Kommentare aus dem erstellten Code gezogen, um daraus eine Dokumentation der Implementierung zu erzeugen.
		
		\item[PlantUML:] Erzeugung von UML-Diagrammen, die Aktivitäten, Aufbau und Funktionen des Systems grafisch darstellen. 
	\end{description}

\subsubsection{Entwicklungswerkzeuge}
	\begin{description}
		\item[Buildsystem:] Es wird \textit{scons} verwendet um eine einheitliche und angepasste Kompilierung von C++-Dateien zu gewährleisten. 
		Hierbei werden sogenannte \textit{sconstruct} erstellt, mit denen verschiedene Vorbedingungen, wie das Einbinden von Bibliotheken und die Auswahl des Kompilers erfüllt werden können.
		
		\item[Entwicklungsumgebung:] Ein beliebiger \textit{Texteditor}, der Syntaxhighlighting für die Programmiersprache C++ unterstützt, wird zum Verfassen des Codes verwendet.
		
		\item[Programmiersprache:] Es wird die hardwarenahe Programmiersprache \textit{C++} verwendet, da diese alle, für das Projekt erforderlichen, Anforderungen erfüllt.
		
		\item[Betriebssystem:] Für den Betrieb der Software ist \textit{Linux} unabdingbar, da das verwendete DPDK-Framework nur auf dessen Grundlage funktioniert.
		
		\item[Bibliotheken:] In diesem Projekt werden viele verschiedene Standardbibliotheken von C++ und vom Auftraggeber bereitgestellte Bibliotheken verwendet. 
		Zusätzlich hierzu wird als spezielles Framework (Ansammlung von Bibliotheken) das DPDK genutzt. 
		Dieses liefert mit seiner umfangreichen Klassensammlung die Grundlage (Verschlüsselung, Forwarding, ...) für das System. 
		
	\end{description}

\subsection{Ergebnisse des Entwurfs für die erste Iteration:}
Die Ergebnisse der ersten Iteration ergeben sich aus dem aktuell zu erreichenden Meilenstein.

\begin{description}
	\leftskip=0,8cm
		\item[Pflichtenheft:] Das Lastenheft wurde vollständig in das Pflichtenheft überführt und um weitere Punkte, im Dialog mit dem Auftraggeber, ergänzt.
		
		\item[Grobentwurf:] Der Grobentwurf umfasst eine erste Übersicht über die Funktions- und Arbeitsweise des Systems.
		
		\item[Implementierung:] Eine erste Lauffähige Implementierung wurde umgesetzt und umfasst die Funktionen eines einfachen Forwardings ohne Verschlüsselung.
		
		\item[Planung:] Es wurden Festlegungen für die weiteren Iterationen getroffen, hierbei wurden insbesondere die Meilensteine für die nächste Iteration festgelegt und das weitere Vorgehen innerhalb des Teams besprochen.
		
	\end{description}

\subsection{Festlegungen für die nächste Iteration:}
Die Festlegungen für die jeweilige nächste Iteration spiegeln sich in den Meilensteinen wieder.

\begin{description}
	\leftskip=0,8cm
		\item[Grobentwurf verfeinern:] Der Grobentwurf wird um spezifischere Diagramme und Beschreibungen zum Feinentwurf ergänzt. Dies geschieht jeweils parallel zu den Meilensteinen, da automatisch jede Änderung dort festgehalten wird.
		
		\item[Implementierung erweitern:] Die aktuelle Implementierung wird um einen ersten Entwurf der Verschlüsselung mit statischem Schlüssel erweitert.
		
		\item[Hardware konfigurieren:] Die zur Verfügung gestellte Hardware wird initial konfiguriert um die Voraussetzungen für das System zu gewährleisten und um die Implementierung ersten Tests unterziehen zu können.
		
	\end{description}
  
\end{document}
