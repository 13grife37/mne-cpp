\section{Results of the design}

The results of the design are outlined in the documentation of design.
There the connections between the different packages, components and classes are explained and visualized via UML-diagrams. 

\subsection{Tools} 

The used tools are software solutions which enable or at least facilitate the different aspects of the organization and development.

\subsubsection{Organization tools}
	\begin{aims}
		
		\item[Sourcecodemanagement:] The source code will be managed via \textit{GitHUB} to ensure a uniform basis of work with the files.
		
		\item[LaTeX:] \textit{LaTeX} is a language for describing text. It is possible to create many different types of documents which afterwards are available in multiple formats. In particular it will be used to create the functional specification, the review documents and the design document.
		
		\item[Doxygen:] With the program the comments will be dragged from the already created code to produce a documentation of the implementation.
		
		\item[Visual Paradigm:] Creating of UML-Diagrams, for the graphically representation of the actions, constructions and functions of  system. 
	\end{aims}

\subsubsection{Developing tools}
	\begin{aims}
		
		\item[Development environment:] Here will be \textit{QtCreator}, for clean code used.
		
		\item[Program language] Here will be program language\textit{C++} used, because C++ meets all requirements of this project.
		
		\item[Operating systems:] The software can be used on Linux, Microsoft Windows and Mac OS.
		
		\item[Libraries:] In this project will C++11 STL and the from employer provided libraries used. 
		We also use QT Libraries with OpenGL, Eingen.
		With there enormous collection of classes  give this libraries the basis for this System. 
		
	\end{aims}

\subsection{Ergebnisse des Entwurfs für die erste Iteration:}
Die Ergebnisse der ersten Iteration ergeben sich aus dem aktuell zu erreichenden Meilenstein.

\begin{aims}
	\leftskip=0,8cm
		\item[Pflichtenheft:] Das Lastenheft wurde vollständig in das Pflichtenheft überführt und um weitere Punkte, im Dialog mit dem Auftraggeber, ergänzt.
		
		\item[Grobentwurf:] Der Grobentwurf umfasst eine erste Übersicht über die Funktions- und Arbeitsweise des Systems.
		
		\item[Implementierung:] Eine erste Lauffähige Implementierung wurde umgesetzt und umfasst die Funktionen eines einfachen Forwardings ohne Verschlüsselung.
		
		\item[Planung:] Es wurden Festlegungen für die weiteren Iterationen getroffen, hierbei wurden insbesondere die Meilensteine für die nächste Iteration festgelegt und das weitere Vorgehen innerhalb des Teams besprochen.
		
	\end{aims}

\subsection{Determinations for the next iteration}

\begin{aims}
	\leftskip=0,8cm
	\item[Refining of preliminary design :] The preliminary design is extended with special diagrams and descriptions to detailed design. This occurs each time  parallel to Milestones, because each change will automatically held.
		
		\item[Further implementation:] Here will the 4 Features implemented and tested.		
		
  
\end{aims}