\subsection{Goals for Third Phase}
In this section the goals for the following phase are outlined. All the features have to be operational and optimized afterwards and the project has to be finished.

\subsubsection{Testing of SCDC}
The correctness of the SCDC-feature has to be tested in order to guarantee realistic results. This could be done by testing some small inputs by hand.
\subsubsection{Testing of Projecting}
If any other data-structure will be used (e.g. some tree-structure), the results can be checked via comparing them to the ones gained by the linear search.
\subsubsection{Portation to MNE Scan}
The MNE Scan application is the front-end in which the features ultimately should be integrated. In this phase we aim to achieve this concomitant with the completion of the features.
\subsubsection{Optimization of the Interpolation}
Interpolating the sensor data is already advanced at this point of time. In the upcoming phase we try to further improve its performance via using compute shaders. This could speed up the calculation as well as it has the advantage of having the data already on the graphics card for further output.
\subsubsection{Bad Channels}
As the devices used to measure the brain activity are not perfect and not every channel is working  (not connected properly, broken, etc.). For this problem a function has to be created, which filters out the so called Bad Channels.
\subsubsection{Fibonacci Heap}
For different cases there are multiple possibilities how to internally implement a heap. A so called Fibonacci Heap could possibly result in better runtimes for the SCDC.