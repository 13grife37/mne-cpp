
%%%DOCUMENTCLASS%%%
\documentclass[a4paper, 12pt, english, fleqn]{article}


%%%USEPACKAGES%%%
\usepackage[utf8]{inputenc}
\usepackage{babel}
\usepackage{coordsys,logsys,color}
\usepackage{fancyhdr}
\usepackage{hyperref}
\usepackage{texdraw}				
\usepackage[T1]{fontenc}					
\usepackage{amsmath,amsfonts,amssymb}	
\usepackage[normalem]{ulem}	
\usepackage{listings}
\usepackage{graphicx}
\usepackage{enumitem}
\usepackage[paper=a4paper,left=40mm,right=40mm,top=35mm,bottom=30mm]{geometry}
\usepackage{titlesec}


%%%PAGESTYLE%%%
\pagestyle{fancy}


%%%LINKING%%%
\hypersetup{colorlinks=true, breaklinks=true, linkcolor=blue, menucolor=darkred, urlcolor=darkblue, citecolor=darkblue}

%%%NEWLIST%%%    Itemnamen bold, Text eingerückt 
\newlist{aims}{enumerate}{1}
\setlist[aims,1]{
	label={Aim~\arabic*},
	leftmargin=*,
	align=left,
	%labelsep=1mm,
	font=\bfseries
}


%%%PARAGRAPHS%%%
\makeatletter
\renewcommand\paragraph{\@startsection{paragraph}{4}{\z@}%
            {-2.5ex\@plus -1ex \@minus -.25ex}%
            {1.25ex \@plus .25ex}%
            {\normalfont\normalsize\bfseries}}
            
\makeatother
\setcounter{secnumdepth}{4} % how many sectioning levels to assign numbers to
\setcounter{tocdepth}{4}    % how many sectioning levels to show in ToC


%%%NEWCOMMAND%%%
\newcommand{\titlefont}[1]{\textcolor{black}{\fontseries{bx}\fontshape{n}\fontsize{30}{0pt} \selectfont #1}}
\newcommand{\titlepagef}[1]{\textcolor{black}{\fontseries{bx}\fontshape{n}\fontsize{14}{0pt} \selectfont #1}}
\newcommand{\gloss}[1]{\textcolor{glossb}{\fontsize{11}{0pt}\selectfont #1}}
\newcommand{\spaceline}[1][8pt]{\vskip #1}
\newcommand{\attrname}[1]{\textcolor{fgcgray}{\scriptsize #1}}
\newcommand{\comment}[1]{\spaceline[5pt] \textcolor{fgcgray}{\scriptsize #1} \spaceline[15pt]}


%%%RENEWCOMMAND%%%
\renewcommand{\familydefault}{cmss}


%%%DEFINECOLOR%%%
\definecolor{fgcgray}{rgb}{0.4, 0.4, 0.4}
\definecolor{darkred}{rgb}{.6,0,0}
\definecolor{glossb}{rgb}{0,0,0.38}


%%%LENGTH SETTINGS%%%
\addtolength{\oddsidemargin}{-1.0cm}
\addtolength{\evensidemargin}{-1.0cm}
\addtolength{\headwidth}{2.0cm}
\addtolength{\textwidth}{2.0cm}

\setlength{\parindent}{0cm}


%%%DEFINITIONS%%%
%Füge Überschriften ein
%Füge Namen ein

\makeatletter

\def\@maketitle{
	%\begin{titlepage}
	
	\begin{center}
		\titlepagef{Software-Project 2017}
		\spaceline
	\end{center}
	
	\begin{center}
		\parbox{\textwidth}{
			\spaceline
			\centering{\titlefont{\@title}}
			\par
			\spaceline
		}
	\end{center}
	
	\begin{center}
		\titlepagef{Real-Time Mesh Utilities}
		\spaceline[2em]
	\end{center}
	
	\begin{center}
		\begin{tabbing}
			Petros Simidyan \qquad \=
			Blerta Hamzallari \qquad \=
			Felix Griesau \qquad \=
			Marco Klamke \\
			Julius Lerm
			\>Lars Debor
			\>Simon Heinke  
			\>Sugandha Sachdeva
		\end{tabbing}
	\end{center}
	
	\spaceline[3em] {
		\begin{flushright}
			\begin{tabular}[t]{rl}
				\attrname{last change:} & \@date
			\end{tabular}
		\end{flushright}
		\par
	}
	\spaceline[5.5em]
	%\end{titlepage}
}

\makeatother
