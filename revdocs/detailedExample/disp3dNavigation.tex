\section{Navigation in Disp3D}

This section describes how to navigate through the 3D control widget of the Disp3D example.

\begin{aims}
	\item[\hspace*{10mm} Where to Navigate] The two new items that use the new features can be found under\\ \textit{sample/Right visual/MEG Data} and \textit{sample/Right visual/EEG Data}.
\end{aims}
	
	
\begin{aims}
	\item[\hspace*{10mm} Turning Data Streams On/Off] Click the checkbox "Stream data on/off" to toggle data streaming.
\end{aims}

\begin{aims}
	\item[\hspace*{10mm} Choosing a Color Map] Click on the second row of text and then choose an entry of the droplist to switch color maps.
\end{aims}

\begin{aims}
	\item[\hspace*{10mm} Configuring Normalization Thresholds] Click on the third row of text and then click left inside the arising graph for the lower threshold and right for the upper threshold.
\end{aims}

\begin{aims}
	\item[\hspace*{10mm} Editing sample delays] Use the spinbox in the fourth row to decrease / increase the delay between visual outputs. 
\end{aims}

\begin{aims}
	\item[\hspace*{10mm} Toggling Data Looping] Click the checkbox in the fifth row to toggle looping of the last received block of sensor data.
\end{aims}

\begin{aims}
	\item[\hspace*{10mm} Sample Averaging] Use the spinbox in the sixth row to decrease / increase the number of samples that should be averaged to one signal.
\end{aims}

\begin{aims}
	\item[\hspace*{10mm} Distance Threshold] Use the spinbox in the seventh row to decrease / increase the distance threshold that is used when running the SCDC and the creation of the interpolation matrix.
\end{aims}

\begin{aims}
	\item[\hspace*{10mm} Interpolation Function] Use the droplist in the last row to choose a function to use during the creation of the interpolation matrix.
\end{aims}