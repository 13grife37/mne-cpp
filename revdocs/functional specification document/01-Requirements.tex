\section{Requirements}
The product receives EEG/MEG sensor data and constructs a real-time 3D visualization of the brains current activity.

\subsection{Mandatory criteria}
The following functions have to be implemented correctly and must fulfill given requirements.
\subsubsection{Surface constrained distance calculation (geodesic problem on meshes)}
The function receives input data, preprocessed by the software environment, structured as a triangulated surface mesh.%vll anders formulieren

%Erklaeren, dass scdc-feature als funktion umgesetzt werden soll

\begin{description}
	\item[C111] Based on that data, the function calculates a matrix that holds values describing the distances between all vertices using double precision. 
	\item[C112] The function must be able to process up to 200,000 vertices.
	\item[C113] The user can limit the calculation to a subset of vertices.
\end{description}

\subsubsection{Point to plane mapping}

%Erklaeren, dass Feature als funktion umgesetzt werden soll

The function receives a set of sensor locations in 3D-Space and maps them onto the underlying surface mesh. Thus every sensor gets assigned to a vertex of the mesh. 

%Erklaeren, dass die punkte schweben, also projektion erforderlich

\begin{description}
	\item[C21] The function must be able to handle data from MEG-sensors which have a known orientation.
	
	\item[C22] The function must be able to handle data from EEG-sensors which are non-orientated.
\end{description}

\subsubsection{Interpolation algorithm}
The algorithm receives a mesh and a subset of vertices %von den letzteren auch die sensorwerte
%Sensorwerte repraesentieren Gehirnaktivitaeten
\begin{description}
	\item[C31] Based on the said subset the algorithm must calculate the values for every vertex of the mesh.
	
	\item[C32] For this, the algorithm creates a matrix storing weights for the later interpolation.
	The interpolation process can be summarized by the following equation $y_{full} = W \cdot y_{sub}$
	, where $W$ is the mentioned matrix and $y_{sub}$ is the current dataset for the known sensors, i.e. vertices.
	
	\item[C33] The calculation of the weight matrix must be based on the result of the SCDC.
	
	%SCDC referenzieren
	
	%+Bad channels beachten
	
\end{description}

\subsubsection{Integration in Disp3D}
	In order to ensure usability within the given framework MNE-CPP, the final visualization must be integrated into the preexisting GUI, namely Disp3D.
	
	\begin{description}
		\item[C41] A new function must be added to the Disp3D tree model. Internally this function must create a new handler.%Lorenz fragen, weil satz im Lastenheft nicht von dieser Welt.
	\end{description}
	
	\subsection{Optional criteria}
	
	%Einfuehrung einige Features wuenschenswert...
	
	\subsubsection{SCDC}
	
	\begin{description}
		\item[C211] The computation time should not exceed 1 second.
	\end{description}
	
	\subsubsection{Interpolation}
	\begin{description}
		\item[C221] One interpolation cycle should take less than 17ms.
		\item[C222] Multiple methods for calculating the weight matrix can be implemented. The user can select one.
	\end{description}
	
	
	\subsection{Differentiating criteria}
	\begin{description}
		\item[C311] The program receives preprocessed data and does get in touch with hardware sensors.
	\begin{description}
		\item[C312] The program does not %bereitet nur daten fuer die visualisierung auf, inhaltlich wird erstmal nichts interpreitert
	\end{description}
	\end{description}



	
